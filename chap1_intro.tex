\chapter{Introduction}

\section{Overview}
The subject of this thesis is the design, construction and initial post-flight analysis of a new atmospheric remote sensing instrument prototype, the Limb Imaging Fourier Transform Experiment (LIFE). This work focuses on a balloon-borne prototype, which will inform the design of a future satellite-borne version. This instrument is designed to take measurements of greenhouse gases in the atmosphere critical to climate change, in the upper troposphere/lower stratosphere (UTLS) region. Constituents and species in this part of the atmosphere must be measured to fill a gap in data and observations regarding this region, which is important for research into climate change. Measurement of the necessary greenhouse gases requires a thermal imaging instrument, a growing technology in the area of atmospheric instrument research. The LIFE instrument is the first thermal imaging atmospheric research instrument developed by the University of Saskatchewan Institute of Space and Atmospheric Science Atmospheric Research Group, and as such will have constraints and design considerations that are different than previous instruments that have operated in the visible and near infrared spectral range. This includes a highly developed and simulated thermal-mechanical model, which is required due to the thermal imaging capabilities of the instrument, and the complexities of the electronics. The instrument also requires a custom infrared detector, which needs to be characterized and optimized for the LIFE instrument application. LIFE is designed following the success of similar thermal imaging instruments such as the Gimballed Limb Observer for Radiance Imaging in the Atmosphere (GLORIA). LIFE is developed as a cost-effective thermal imaging instrument to allow collaboration and the confirmation of measurements taken by GLORIA. Overall, LIFE is designed as a new instrument prototype to take measurements of greenhouse gases, allowing better knowledge of climate change and the atmosphere.

There are many reasons why it is important to measure and understand the atmosphere of Earth. The atmosphere, most importantly, is critical to all life on Earth. Without the atmosphere, water would not exist on the surface, life would not be shielded from dangerous cosmic radiation, and the average temperature would be drastically lower. It is important, therefore, to measure how it is changing, and specifically the impact of human activity. Climate change, along with other important atmospheric phenomena such as weather patterns, are all researched through atmospheric models. These models allow prediction of weather, which is important to prepare and mitigate for serious weather effects such as natural disasters. For climate change, it is crucial to understand the effects of pollution in the atmosphere to know the damage caused and what needs to be done to mitigate this. To allow these models to be used in these applications, they need to be as accurate as possible. This requires up-to-date measurements of the atmosphere and its constituents, such as greenhouse gases and aerosols. They also need to be known at all atmospheric layers as the vertical distribution changes the climate sensitivity.

There are many forms of atmospheric instrumentation. Some are launched as satellites, some are implemented as ground-based systems, and many between such as aircraft or balloon-mounted instruments. The majority of these instruments are passive, or \textit{remote sensing}, which gather data without emitting any radiation from the instrument. Data is imaged by looking at the radiation from the atmospheric constituents or the effect they have on radiation from other sources. For all non-ground based instruments, there are two main types: nadir-sounding instruments and limb-sounding instruments. Nadir-sounding instruments take measurements by looking directly down towards Earth, while limb-sounding is when the instrument examines the \textit{limb} of the atmosphere, or tangentially through the atmosphere towards space~\citep{SPARC}. Most instruments that gather data on the atmospheric region of interest to this thesis use the limb-sounding method.

Limb-sounding instruments can be classified depending on the source of the measured radiation: solar occultation, stellar occultation, limb scattering, and limb emission. Limb emission works by measuring radiation emitted by the atmosphere, either thermally or photochemically, along the instrument line of sight (LOS). A problem with this method is that these are generally low signal level measurements, but can be detected with sensitive instruments. This method does not rely on the sun as a source of radiation, so measurements can be made during day or night. Limb scattering instruments measure photons originating from the sun that have been scattered from the atmosphere. This does not require very sensitive instruments but measurements can only be taken in sunlight. Solar occultation measurements look through the atmosphere directly at the sun and measure the spectral attenuation of the solar irradiance. This requires measurements to be taken with the LOS towards the sun. Stellar occultation is similar to solar occultation, however the stars are used as a radiation source rather than the sun, which allows a longer measurement window~\citep{SPARC}.

As technology improves and more sensitive detectors are readily available, limb emission and limb scattering methods are becoming more popular. With a complete independence on source for limb emission and only sunlight necessary for limb scattering, measurements can be taken much more often and with less constraint than with other methods. Limb emission in particular has no viewing angle requirements, and can be chosen freely, as long as it is well known to prevent data propagation errors~\citep{IR_limb_emission_measurements}~\citep{SPARC}. This allows a large vertical range, so more of the atmospheric region of interest can be measured. This method is used by LIFE, as well as similar instruments such as the Michelson Interferometer for Passive Atmospheric Sounding (MIPAS) and GLORIA. A common remote sensing device used in limb emission instruments is the Fourier Transform Spectrometer (FTS). FTS systems have a high sensitivity compared to other spectrometers, and can be used with an infrared detector to allow thermal emission measurements. They also have the ability to capture high spectral resolution over a wide spectral range. Almost all thermal limb emission imaging instruments utilize an FTS with an infrared detector to take measurements~\citep{GLORIA_concept}~\citep{MIPAS_instrument}. 

A key early instrument that utilized an FTS to measure limb emissions from space was the MIPAS instrument. It provided atmospheric data on temperature, trace species and cloud distributions. The overall goal of the instrument was to observe global changes in the composition of the atmosphere resulting from pollution and other man-made effects. On-board the EnviSat satellite developed by the European Space Agency, it provided profiles of H\textsubscript{2}O, O\textsubscript{3}, CH\textsubscript{4}, N\textsubscript{2}O, HNO\textsubscript{3}, and NO\textsubscript{2}~\citep{MIPAS_instrument}. However, a limitation of this instrument, and similar instruments at the time such as the Atmospheric Chemistry Experiment FTS (ACE-FTS), was that it only had a single detector pixel~\citep{SPARC}~\citep{ACE_conference}. To cover a larger field-of-view (FOV) and measure the atmospheric limb, limb scanning was used, where the instrument moves the LOS upwards and downwards. Eventually, a new type of FTS, known as an Imaging Fourier Transform Spectrometer (IFTS) allowed multiple pixels to image through the FTS at once, removing the need for limb scanning, and creating a better and more uniform image. The first instrument to utilize this technology was GLORIA.

GLORIA is an airborne limb imaging instrument operating in the thermal infrared region, and is conceptually similar to the LIFE instrument that is the focus of this thesis. GLORIA consists of an IFTS device mounted on a gimbal and provides high spatial and spectral resolution measurements of the upper troposphere/lower stratosphere (UTLS) region~\citep{GLORIA_concept}~\citep{GLORIA_PhD}. GLORIA utilizes a two-dimensional pixel array to obtain this resolution, as no spatial scanning of the imager is necessary. The detector is sensitive to numerous species, including H\textsubscript{2}O, O\textsubscript{3}, CCl\textsubscript{4}, HNO\textsubscript{3}, ClONO\textsubscript{2}, HO\textsubscript{2}NO\textsubscript{2}, and CFCs. With the high spatial resolution, GLORIA will measure the steep gradients in trace gases and characteristics of clouds in the UTLS region, as well as provide insight into the stratosphere-troposphere exchange (STE) that has been affected by climate change and plays an important role in climate models~\citep{GLORIA_PhD}~\citep{GLORIA_objectives}.

The instruments described above, and any that are operating at atmospheric altitudes, have many design constraints that must be considered as a result of this harsh environment. An important part of the LIFE design and key part of this thesis was considering these constraints and ensuring that it would survive and still be able to take optimal measurements. The thermal environment at these altitudes has a number of considerations that may not have to be considered for either ground based or space based systems. Temperatures during ascent can reach as low as -50°C, and instruments that are operated during daylight can heat drastically due to the sun. This is of particular importance to thermal imaging instruments, as drastic temperature changes in the instrument can effect the data through self-emission, the imaging of instrument optics. Thermal control measures must be used, and thermal simulations are important to instrument survival. This thesis describes the design process for LIFE, which relies heavily on the thermal requirements and constraints placed on the instrument to develop an operational instrument.

\section{LIFE}

LIFE is designed to use new imaging Fourier Transform Spectrometer (IFTS) technology to image the atmosphere in the thermal regime. It is the second atmospheric instrument to be developed around an IFTS, following the successful development and operation of the GLORIA instrument from the Karlsruhe Institute of Technology (KIT). The improvement in technology of GLORIA and LIFE over previous atmospheric thermal imaging instruments, such as MIPAS, is the vertical imaging capabilities with the IFTS. Previous FTS based instruments took atmospheric images through the use of a single pixel scanning the atmosphere; The use of a pixel array allows single images to be taken and avoid the need for scanning~\citep{GLORIA_objectives}. LIFE is designed in the footsteps of GLORIA, which aims to meet the capabilities of GLORIA while creating an instrument that is less expensive and has a smaller footprint. 

The instrument is designed to measure trace greenhouse gases in the UTLS region. These greenhouse gases play a critical role in climate change, and it is important that information is gathered to inform climate models. There is a gap in knowledge of key greenhouse gases in this region of the atmosphere, and the LIFE and GLORIA instruments aim to close that gap and provide measurements on levels of various constituents in this region. Some of the important greenhouse gases that are measured by the LIFE instrument are as follows: H\textsubscript{2}O, O\textsubscript{3}, N\textsubscript{2}O, and CH\textsubscript{4}. Operating in a similar spectral range to GLORIA, it is designed to measure in the wavenumber region of 700 cm\textsuperscript{-1} to 1400 cm\textsuperscript{-1}. It will take these measurements from the lower stratosphere, at an altitude of 35 km. The instrument will image vertically from this altitude down to an altitude of 8km~\citep{ethans_thesis}. This contrasts to the GLORIA instrument, which took measurements from an aircraft at lower altitudes.

The first version of the instrument, as developed in this thesis, is a prototype designed to fly on a high-altitude balloon at the aforementioned altitude of 35 km. As a prototype it is designed to demonstrate that a IFTS based thermal imaging instrument can be developed and take good measurements for a reasonable cost and size. It will inform future designs of the instrument, eventually leading to a satellite based design. The initial development of the LIFE instrument, including the core optical design and the initial modelling of the optical system, was done by Ethan Runge for his MSc. thesis. This thesis discusses two core tasks of the development of this core prototype: The thermal-mechanical design, and the characterization of the infrared detector.

\section{Outline}
This thesis discusses the thermal-mechanical design of the first balloon-borne prototype of the LIFE instrument, as well as the characterization of the MCT infrared detector. Chapter \ref{bkgnd} presents background for the rest of the thesis. A discussion is given for limb imaging in the UTLS, and previous instruments that are the inspiration for the LIFE instrument. Background on the thermal regime of the balloon flight and environment is also discussed, including thermal phenomena, thermal controls and the thermal designs of similar instruments. Finally, this chapter also contains background on different types of infrared detectors, why the MCT detector was chosen, and issues to be characterized.

One of the two main aspects of this thesis is the thermal-mechanical design, which is discussed in Chapter \ref{thermal}. This chapter discusses the requirements for the thermal design, both for the optical system and the electronics. It also discusses the mechanical requirements for the instrument and the flight on-board the National Centre for Space Studies (CNES) gondola. The thermal environment is described in more detail, and the operations of the software used for the simulations is discussed. The majority of this chapter is the process of the thermal and mechanical design of the LIFE instrument, through a variety of iterations, simulations and environments.

Following the flight of the LIFE instrument in the late summer, the thermal model was compared to temperatures seen in flight. This is discussed in Chapter \ref{postflight}. It also discusses the building of a full flight model for future atmospheric flight instruments for all stages of the flight. This section also covers the results of the flight more generally, including the campaign and mechanical results.

The infrared (IR) detector in the LIFE instrument needed to be characterized with settings chosen for optimal measurements. The process of this characterization is described in Chapter \ref{detector}. This involves numerous measurements and testing to ensure proper operation and knowledge of the detector. The detector itself is described in detail, as well as the results of this characterization.

Chapter \ref{future} goes into detail on the future work necessary for LIFE and the thermal model of the instrument. Thermal-mechanical changes based on what was seen during flight are discussed, continuing characterization of the MCT detector, as well as recommended updates to the atmospheric instrument thermal model.