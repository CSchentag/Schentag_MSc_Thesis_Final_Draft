\chapter{Post-Flight Analysis} \label{postflight}

\section{Overview}
This chapter discusses the post-flight analysis of the thermal-mechanical model of LIFE. In the first section, an overall summary of the stratospheric balloon flight campaign in Timmins, Ontario is presented, including work done while in Timmins and the overall results of the flight. The next section describes the mechanical results, in terms of how the mechanical design performed and some modifications for the future. The rest of the chapter discusses the creation of a full flight thermal profile of the instrument. Split into four subsections, different parts of the flight are examined in terms of the thermal results, and a thermal model is created to match these results. This will help to inform future instrument thermal models, as very little information on the flight environment is currently available. All post-flight analysis regarding the MCT Detector is presented in the next chapter.

\section{Flight Campaign \& Results}
The campaign took place over three weeks in Timmins, Ontario, at the CSA/CNES high-altitude balloon base. The work required prior to flight included initial checks of the instrument, ensuring proper operation after transport. Once this was complete, the instrument was integrated onto the gondola. This required ensuring that it could be properly fastened to the gondola deck, and that it would operate properly while connected to the gondola power supplies and computers. An image of the instrument integrated into the gondola is presented in Figure \ref{fig:LIFE_ON_GONDOLA}.

\begin{figure}
    \centering
    \scalebox{1}[-1]{\includegraphics[width=0.7\linewidth]{chap4_images/LIFE_on_gondola.jpg}}
    \caption{LIFE integrated into the CNES gondola.}
    \label{fig:LIFE_ON_GONDOLA}
\end{figure}

The integration and checks all went well. The instrument operated properly both through initial tests and while connected to the gondola. With this completed, the instrument was ready to launch, and would wait until the weather was clear and the CSA was ready to launch. This occurred on August 31st. The instrument was loaded onto the gondola at 7pm, initial checks and boot up was completed, final closures and connections to the instrument were made, and the instrument launched at 10pm local time. An image of the instrument sitting on the gondola waiting for launch and the inflation of the gondola balloon is shown in Figure \ref{fig:LIFE_BEFORE_LAUNCH}.

\begin{figure}
    \begin{subfigure}[h]{0.49\textwidth}
        \centering
        \rotatebox[origin=c]{270}{\includegraphics[width=\linewidth]{chap4_images/LIFE_on_gondola_2.jpg}}
        \caption{LIFE on gondola, on launchpad prior to launch.}
        \label{fig:LIFE_on_launchpad}
    \end{subfigure}
    \begin{subfigure}[h]{0.49\textwidth}
        \centering
        \includegraphics[width=\textwidth]{chap4_images/gondola_balloon.jpg}
        \caption{The gondola balloon being inflated shortly before launch.}
        \label{fig:gondola_balloon}
    \end{subfigure}
    \caption{LIFE being prepared for launch.}
    \label{fig:LIFE_BEFORE_LAUNCH}
\end{figure}

The instrument reached a float altitude of approximately 36\,km at 1am. Measurements were taken throughout the ascent, but the science quality observations were made from stable float altitude. The instrument operated without issue through the entire float phase. At about 5:30am, the sun rose, and the instrument began to warm. Initially, the CSA had no plans to operate the instrument in daylight, with the gondola to be brought down prior to sunrise or shortly after. With this as the expected plan, LIFE was shutdown at around 6am. However, CNES was having trouble finding a proper landing location that was not close to any water, which lengthened the flight. As the flight was extended, LIFE was rebooted, so that more measurements could be taken and thermal data could be taken while the sun rose. The sun has no effect on the physics of the instrument, and as long as it is not directly looking at the sun the instrument can be operated at any time. The decision was made to operate the instrument until components reached their temperature operating limits. Normally, if sunlight was expected, this would not be as much of an issue. CNES, if the gondola will fly during the day, installs sun-shields to shade the instruments to minimize the effect of the sun, but as this was planned as only a night flight these were not added. Thus the sun shining directly on the instrument contributed significantly to the heating of the instrument. The flight continued well into the morning, only coming down shortly after noon. Thermal data and measurements were successfully taken for an extra 6 hours past sunrise. This provided more data to attempt to model what the instrument would see when the sun rose, as is described later in this chapter. The instrument landed safely in the early afternoon. An image of the instrument at its landing sight is shown in Figure \ref{fig:gondola_after_landing}. 

\begin{figure}
    \centering
    \rotatebox[origin=c]{270}{\includegraphics[width=0.5\linewidth]{chap4_images/gondola_after_landing.jpeg}}
    \caption{Gondola after landing.}
    \label{fig:gondola_after_landing}
\end{figure}

The instrument was recovered and was brought back to base by the next morning, and the team left the base shortly afterwards. All images and measurements for the instrument were taken successfully, and thermal data of the instrument was successfully saved so that an accurate thermal model of the instrument based on this data could be created.

\section{Mechanical Results}
Although the flight overall was successful, the landing did not go as planned. For all gondola flights, the landing is where the gondola has the greatest shock, with a force of typically around 10-15\,g. This force is designed for and tested in the CSA mechanical verification spreadsheet, as discussed in Section \ref{Mech_changes}. However, out of the planned three parachutes used during descent, only two opened properly. This caused two problems: one was that the gondola was descending much faster than was planned for. The second was that in addition to the descent speed, the gondola was falling at an angle, due to the three parachutes normally forming a triangle. Without the third, the weight was offset, and the edge of the gondola (towards the LIFE side) hit the ground first, rather than the bottom of the gondola which was cushioned for the landing. The gondola eventually hit the ground with a shock that saturated the on-board force sensors at 20\,g.

This unplanned force did not cause any significant structural damage. Even though the mechanical interfaces were tested to 15\,g, the safety factor helped the instrument survive over 20\,g. However, there was some evidence of the impact. The largest was that the bolts holding the Electronics Box to the instrument baseplate were stretched out, meaning that the interface was very close to breaking. This would have caused the LIFE instrument to be totally destroyed, as the crucial connections between the electronics in this box and the detector in the Optics Box would have been destroyed as well. An image of the result of the impact on this interface is shown in Figure \ref{fig:landing_damage_image}.

\begin{figure}
    \centering
    \includegraphics[width=0.8\textwidth]{chap4_images/LIFE_impact_result.jpg}
    \caption{A gap between the Electronics Box and baseplate, due to bolt stretching caused by the shock of landing.}
    \label{fig:landing_damage_image}
\end{figure}

This is one of the first things that needs to be fixed prior to LIFE being used again. However, this was not the only damage caused. When the instrument was taken back to the university lab from Timmins to perform post-flight tests, it was found that the detector could not reach the necessary measurement temperature of -198°C, only reaching -185°C. In addition, although images taken at -185°C are noisy but often usable, the images taken during these tests were much too noisy to be useful. There was an issue with the detector as a result of the landing impact. Through some data analysis on the noise during flight, and eventually removing the detector to send it to the manufacturer, it was discovered that the cold stop of the detector was now no longer attached to the Stirling cooler. Thus it was not being cooled properly and could not be used. There is a very high cost in repairing the detector so that the instrument may be able to be used or modified for further missions, and it is unsure whether is will be fixed. As a result, no post-flight tests could be done to perform further verification and testing of the instrument.

\section{Flight Temperature Model}\label{flight_temp_model}
After the flight was completed, the temperatures measured during flight were compared to the model. The most important aspect from the thermal-mechanical design was that all components stayed within the required temperature limits for the float part of the flight, when everything was operating as needed. However, the temperatures did not exactly match the measurements made pre-flight in the TVAC chamber, mainly because there are so many variables in high-altitude balloon flights. It is difficult to make an accurate model, even for just the simplest parts of the flight, because there are a number of variables that may effect the temperature, and next to none are documented or studied. For example, there is no information on how convection changes as altitude increases, especially at altitudes above the troposphere. There is no information on the precise amount of solar heating on an instrument that is dependent on altitude. A number of other questions also remain.

In this part of the thesis, a thermal model of LIFE was created for the entire balloon flight. This includes all phases, not just the float, which is the easiest and was modeled prior to flight. This model could then be a basis for future high-altitude balloon instruments to draw from, for thermal simulations and what to expect during a flight. This will help to ensure better survivability in future instruments.

Of course, there are too many variables to use this thermal model for all future instruments in entirety. It is meant to provide a starting point, and can help to plan for what to expect beyond ensuring that it will survive during the float phase. It can help to plan for the temperature decreases seen through the ascent, and help to design for better protection of the instrument should it be running in daylight. And through future missions, the model can be improved through new thermal data, until a thermal model with multiple sets of temperatures and improvements has been made can be used for future instruments with little changes. This will help to reduce the workload on future instruments and reduce costs and engineering time. 

Before going further into the temperature simulations, the flight temperature profile as a whole is examined in more detail. A plot of the flight temperatures measured by both the LIFE temperature sensors and a temperature sensor on-board the gondola baseplate is shown in Figure \ref{fig:full_temps_no_sims}. The altitude of the gondola is also shown in this figure.

\begin{figure}
    \centering
    \includegraphics[width=\textwidth]{chap4_images/Flight_temp_values_no_sims_V2.png}
    \caption{Temperatures over the course of the August/September 2019 Timmins flight.}
    \label{fig:full_temps_no_sims}
\end{figure}

Described here at a high level, with more detail in subsequent sections, the different phases of the flight are seen here. The initial linear increase in temperature is from the instrument heating up as it sits running and waiting for launch on the launchpad. It increases quickly due to the lack of fans, which needed to be turned off and covered prior to flight. Immediately following launch at 2:00 UTC, the temperatures drop dramatically. This is the result of the cold region of the tropopause quickly cooling the instrument; on the night of the LIFE launch, the tropopause region was at approximately -50°C, which the gondola travelled through for almost half an hour. The instrument begins to warm once it reaches the warmer stratosphere. The gondola stabilizes just above 35km at roughly 5:00 UTC, where measurements are taken.

Through the main measurement phase of the flight, from 5:00 to 10:30 UTC, the temperatures are extremely stable. The required temperature drift is maintained through this time frame, only changing once the instrument begins to heat from the sunrise. All temperatures are also within required limits, with the optics staying just below 15°C and the electronics staying in the 5-10°C range. This is slightly cooler than expected, which is due to the initially cold temperatures from the ascent. At the end of this phase, the sun begins to rise, and temperatures begin to rise as well. A small dip shows where the instrument was momentarily turned off, and the flight was expected to end. When it was turned back on, the instrument had cooled slightly, but began to rapidly warm due to the electronics and sunlight, until everything was finally turned off at 14:30 UTC. 

The goal for the thermal model is to match the simulated temperatures to what is seen in Figure \ref{fig:full_temps_no_sims} through a series of iterations. For the purposes of the simulation, the flight was split into three major phases: The ascent through the troposphere and tropopause, the float period when the instrument is at altitude and taking measurements, and the sunrise, when the sun begins to shine on the instrument and have a major effect on the temperatures. Each of these simulations were completed separately, and the final temperatures of each simulation made to match precisely what was seen during flight. The final temperatures of a simulation would be used as the initial temperatures of the next simulation, so it was important that they were very close to the actual temperatures to avoid propagation errors. Once the final temperatures were met, the temperature curve of each section was compared to the temperature curve of the flight to ensure it fit.

\subsection{Ascent}\label{ascent}
The first part of the flight was the ascent. This included the rise from ground level, up through the tropopause, and reaching the float level of 35km in the stratosphere. This was the most difficult part of the flight to simulate, due to having the most variables and the most significant change in temperatures. Convection only plays a role in this part of the flight, and is the biggest unknown that will need to be determined. Radiation, conduction, and forced convection as a result of the speed of the rising balloon gondola all need to be considered for this part of the simulation. As a result, it is likely the least accurate part of the entire thermal model, and also required the most iterations to model correctly.

\subsubsection{Measured Temperatures}
The simulation was split into two halves: The ascent up until the troposphere, and the ascent past the troposphere up to the stratosphere. The centre of this split is the temperature minimum for all components. The time of the launch was 2:00 UTC. The time until the temperature minimum, or when the gondola left the tropopause and began to warm up again, was approximately 3:00 UTC. The second part of the ascent, which ended when the gondola stabilized at the float altitude of 35km, was from 3:00 to 5:00 UTC. The temperature measurements of the entire ascent phase of the flight along with the altitude during this time is shown in Figure \ref{fig:ascent_temps_no_sims}.

\begin{figure}
    \centering
    \includegraphics[width=\textwidth]{chap4_images/ascent_images/Ascent_temps_no_sims_V2.png}
    \caption{Flight temperatures through the ascent, the first three hours of the flight.}
    \label{fig:ascent_temps_no_sims}
\end{figure}

For the hour leading up until launch, the instrument is sitting on the launchpad, powered on and waiting; this is where the temperatures are slowly increasing. The most important part of this phase is the temperature drop shortly after launch. The cold air has the largest affect on the gondola baseplate, which drops to -40°C by the end of the tropopause, and overall it drops 50°C in as little as half an hour. This is the result of the cold air and the forced convection at these speeds and altitudes. As the LIFE temperature sensors are inside the boxes, the temperature changes are slightly delayed. The effects of the isolation of the Optics Box is evident here. The three sensors that show the largest and quickest temperature drop are in the Electronics Box and the Blackbody Electronics Box, which are not insulated from the baseplate. The lowest temperature seen anywhere on-board LIFE is just below 0°C, where the top of the Electronics Box dips due to the effect of convection on the largest open plate in the instrument. This was one of the few parts of the instrument that dropped outside of its required temperature range, but quickly warmed again from the heaters powering on, and no cold damage was sustained. Once the environmental temperature warmed and the heaters fully powered on, the temperatures quickly increased back to nominal.

All optics components show a much slower temperature decrease, as they are slowly cooled through the two layers of thermal insulation of the titanium spacers. Instead of dropping quickly and warming from the heaters, these components slowly decrease until they begin to stabilize from the warmer temperatures outside the box and the optics plate heater maintains the required temperatures inside the box. These temperatures were monitored closely through this phase of the flight to ensure temperatures were not dropping too quickly and the temperatures were being maintained towards the end of the ascent.

\subsubsection{Simulations}
With the ascent temperatures fully described and understood, a thermal model was then developed to attempt to match these temperatures. All thermal loads are described here, and the decisions and iterations behind each. Through 30 iterations for the first half of the ascent and another 31 for the second half, each thermal load was examined and tweaked. There are a total of 41 thermal loads in these simulations.

An important and difficult aspect of this first stage were the initial temperatures. Because the instrument had been running on the launchpad for an hour prior to launch, different components were at different temperatures as a result of the electronics. For example, the MCT detector dumps a large amount of heat as it cools, and this was sitting at above 30°C as the instrument was launched. In a SolidWorks thermal simulation, it is very difficult to choose initial temperatures for different components. Most often, the entire instrument has one initial temperature, and if not then each component of the entire assembly must have its own initial temperature. In the LIFE model, this is upwards of 1000 components. One of the motivations for creating the ascent temperature model was to provide accurate initial temperatures for the float component of the flight without having to choose each of these separate initial temperatures. To avoid having to simulate the instrument sitting on the launchpad to get these initial temperatures, an average of the core component temperatures was taken, which was 23°C. This temperature was then applied to all components. 

Conduction is the easiest heat transfer property to model, as the environment has no effect, unlike radiation or convection. Discussed previously in Section \ref{final_pre_flight_sims}, the biggest unknown with conduction is the thermal conductivity across mechanical interfaces. A large aspect of the TVAC simulations and thermal comparisons were determining the actual values for these conductivites. Through these tests the conductivity for the boxes to the baseplate (two anodized surfaces) was found to be 0.16\,$\mathrm{W/m^2K}$, and the conductivity for the optics components to their mounts, and their mounts to the baseplate (anodized to bare aluminum surfaces), was 0.20\,$\mathrm{W/m^2K}$. These conductivites were not changed for the flight, and are used in all flight simulations.

Radiation from various surfaces of LIFE is a large problem in the simulations. To be able to simulate radiation, three values are needed: The view factor to other surfaces, the emissivity of the surface, and the temperature of nearby surfaces. The emissivity is the most well known property of these three values, and does not need to be tweaked. The three most common materials and surfaces all have well known emissivities: Anodized aluminum, which is the majority of the box parts as well as the optics breadboard, has an emissivity of 0.77. For circuit boards, the emissivity of silicon is 0.6. Bare aluminum, such as outer surfaces of the blackbody assembly, has an emissivity of 0.05. 

The other two properties needed for radiation are much more difficult to determine. However, the view factor was slightly easier as it stayed constant through all simulations, so once good values were found in the initial ascent phase simulations they did not need to be changed again. As described in Section \ref{radiation_sec}, the view factor can be calculated automatically through the simulations, but due to the complexity of the view factor integral equation, it dramatically increases the solve time. It is quicker if the values are estimated and entered manually. Each radiative surface must be assigned a view factor to another surface at a certain temperature, and the view factor is a number representing the percentage of the surface that can be seen, from 0 to 0.99. Some of these estimations were simple, such as the outer surfaces of the boxes to the inner gondola walls. As there was nothing obstructing the view between these two surfaces, they would have a high view factor, above 0.9. Similar estimations could be made between the electronics surfaces in the Electronics Box to the box wall.

However, many surfaces proved more difficult. For example, a difficult surface to estimate was the optics breadboard. Different parts of the breadboard are in view of different components, such as the Optics Box wall, the FTS, or the detector. The best estimation was to chose the component that had the highest view factor, which for the breadboard would be the wall of the Optics Box. Similar decisions needed to be made for interfaces between box walls, and the inner surfaces of boxes where electronics are mounted.

The most variable component of the radiative heat loads was the ambient temperature. For all previous tests, such as the TVAC and initial test simulations, these temperatures were constant, as the environmental constraint temperatures were held constant. However, with the rapidly changing environment for the ascent, the ambient temperatures also rapidly change. Thus, a temperature curve must be created and input into SolidWorks. This had to be done for all major components, and the curve was approximately modelled after what was seen during flight, and the known temperature of the atmosphere at increasing altitudes. This could be accomplished easiest for the outer surfaces of the instrument, especially those that viewed the atmosphere, as this temperature was well known. More complex were the outer surfaces of the instrument that viewed the insulated walls of the gondola, which were a reflective insulation material which would have different temperature effects. The ambient temperatures for interior components and surfaces needed to be chosen based on both measured flight temperatures and what the temperatures were from previous iterations of the simulations. One of the most difficult parts of choosing these temperatures was that if the ambient temperature was changed, to reflect a temperature change from the latest version of the iteration, it could have rippling effects causing more temperature changes in future iterations. In short, changing the ambient temperature of one part to reflect another could change the ambient temperature of that component, and it would need to be updated again for the next iteration. This is one reason why so many iterations were necessary. An example ambient temperature curve is shown in Figure \ref{fig:ascent_pt1_top_box_rad}, for the first hour of the ascent for the external box surfaces. A curve like this is created for almost every radiating component in the model.

\begin{figure}
    \centering
    \includegraphics[width=0.7\textwidth]{chap4_images/ascent_images/ascent_pt1_top_box_rad.png}
    \caption{An example of creating the ambient temperature curve for the radiation of a component, specifically for the top surfaces of the boxes for the first half of the ascent.}
    \label{fig:ascent_pt1_top_box_rad}
\end{figure}

Finally, the last remaining and likely most difficult heat load is convection. This is the most complex and hardest part to model because it has two unknown features, that both change with time. The first is the convection coefficient, i.e. how quickly heat is flowing to or from the surface as a result of the convection. It is extremely difficult to calculate, and even if a value is calculated the error is much too large to be able to use confidently. In addition to the convection coefficient, the ambient temperature must also be varied throughout the simulation, in the same way as it is for radiation. This value was at least partially known, through the temperature sensors from flight and from the known temperatures of the atmosphere. The convection coefficient however would be roughly estimated and changed numerous times.

The convection coefficient of air can be anywhere from 1-10 $\mathrm{W/m^2K}$ for free flowing air, and anywhere from 1-100 $\mathrm{W/m^2K}$ for forced convection. It also drastically changes with pressure, as there is less density of fluid to be able to transfer heat. As discussed in Section \ref{convection_sec}, there are very few studies of the effect of pressure on convection, and only for the purposes of high heat industrial processes. As there was no information for a high-altitude balloon scenario, information was taken from the forced convection scenario, and steadily decreasing as a result of the decreasing pressure. For the exterior parts of the instrument, forced convection would be the most prevalent. However, as the gondola was mostly covered, the forced convection would not be overly strong. An estimation of 10 $\mathrm{W/m^2K}$ was made as an initial value, and steadily decreasing to zero by the end of the first half of the ascent. The pressure past the tropopause is estimated to be too low for convection to have any effect past this point. Convection was estimated inside the boxes to be between 1-5 $\mathrm{W/m^2K}$, as the enclosures would limit any forced convection due to moving air from the ascent. Overall, these convection loads only played a part in the first part of the ascent, with the exception of some small convection on the bigger surfaces for the second part of the ascent. The ambient temperatures were kept the same or similar as the values used for radiation, to make the iterations simpler and for continuity across all components.

\begin{figure}
    \centering
    \begin{subfigure}[h]{0.9\textwidth}
        \centering
        \includegraphics[width=\textwidth]{chap4_images/ascent_images/ascent_pt1/Test_30_BBEbox_FIXED.png}
        \caption{Thermal model of LIFE instrument after the first hour of flight, Blackbody Electronics Box view.}
        \label{fig:ascent_pt1_model_bbebox}
    \end{subfigure}
    \begin{subfigure}[h]{0.9\textwidth}
        \centering
        \includegraphics[width=\textwidth]{chap4_images/ascent_images/ascent_pt1/Test_30_Ebox_FIXED.png}
        \caption{Thermal model of LIFE instrument after the first hour of flight, Electronics Box view.}
        \label{fig:ascent_pt1_model_ebox}
    \end{subfigure}
    \caption{LIFE thermal model following first half of the ascent.}
    \label{ascent_pt1_model}
\end{figure}

\begin{figure}
    \centering
    \begin{subfigure}[h]{0.9\textwidth}
        \centering
        \includegraphics[width=\textwidth]{chap4_images/ascent_images/ascent_pt2/Test_30_BBEbox_FIXED.png}
        \caption{Thermal model of LIFE instrument at the end of the ascent, Blackbody Electronics Box view.}
        \label{fig:ascent_pt2_model_bbebox}
    \end{subfigure}
    \begin{subfigure}[h]{0.9\textwidth}
        \centering
        \includegraphics[width=\textwidth]{chap4_images/ascent_images/ascent_pt2/Test_30_Ebox_FIXED.png}
        \caption{Thermal model of LIFE instrument at the end of the ascent, Electronics Box view.}
        \label{fig:ascent_pt2_model_ebox}
    \end{subfigure}
    \caption{LIFE thermal model, as the instrument is reaching the float altitude.}
    \label{ascent_pt2_model}
\end{figure}

There are too many components and loads to be able to accurately describe all values chosen and all iterations here. A list of all loads, components and plots can be automatically downloaded from SolidWorks after a simulation, and this is added as an appendix for the final simulation, in Appendix \ref{post_flight_thermal_properties_appendix}. After a total of 61 iterations, the final temperature models for the end of the first phase of the ascent and the end of the second phase of the ascent are shown in Figures \ref{ascent_pt1_model} and \ref{ascent_pt2_model}, respectively. A plot showing the simulated temperature curves of the thermal model, measured at the same locations as the on-board temperature sensors, is shown in Figure \ref{fig:ascent_temps_with_sims} as a verification of the model.

\begin{figure}
    \centering
    \includegraphics[width=\textwidth]{chap4_images/ascent_images/Ascent_temps_with_sims_V2.png}
    \caption{Flight temperatures through the ascent, compared to the simulated temperature curves after model updates. The corresponding actual and simulated temperature data is the same colour, with the simulated data shown dashed.}
    \label{fig:ascent_temps_with_sims}
\end{figure}

From the plot in Figure \ref{fig:ascent_temps_with_sims}, the model now fits the ascent very well, with the largest error being 1°C. Errors in the first part of the ascent are due to the generalized initial temperatures, but they begin to match the actual temperatures early in the ascent. It was ensured that the temperatures were as close as possible to the actual temperatures at the end of the simulation, so the initial temperatures of the next phase are as close as possible. The error between actual and simulated temperatures for the end of the ascent are within 0.2°C for the critical components, which is well within the goal of 1°C.

\subsection{Float}
The next phase of flight was the \textit{float} phase, or the phase where the instrument stayed steady at 36km altitude, operated nominally and took measurements. This was the simplest part of the flight to simulate, as the environmental effects were constant throughout this stage, and convection no longer played a role. Some transient temperatures are still used for some ambient radiation temperatures, but for the most part all heat loads stay steady. This was also the part of the flight that was simulated in the TVAC chamber, and as a result no issues were expected. This stage continues until the sun rises.

\subsubsection{Measured Temperatures}
The beginning of this phase of the flight was when the ascent was officially over, and the gondola had stabilized at the required altitude. This occurred at 5:00 UTC. The sun rose just after 10:00 UTC, and began to have an affect around 10:30 UTC. When the solar flux needed to be included, a new simulation was created. This was characterized as the \textit{sunrise} phase of the flight, and is described later. The temperature measurements of the float phase, along with the gondola altitude, are shown in Figure \ref{fig:float_temps_no_sims}.

\begin{figure}
    \centering
    \includegraphics[width=\textwidth]{chap4_images/float_images/Float_temps_no_sims_V2.png}
    \caption{Flight temperatures through the float phase, five and a half hours where most measurements were taken.}
    \label{fig:float_temps_no_sims}
\end{figure}

The temperatures fully stabilized around 6:00 UTC. After this point, the temperatures remain very stable, until the gondola deck begins to heat at around 10:30 UTC. The warmest component was the MCT detector mount, which was expected as the detector was dumping heat to its surroundings. All optics were kept within a couple degrees, which is ideal. The double isolation of this plate helped to keep the temperatures constant across the entire plate, which will help to remove self-emission from the resulting data. In addition, the temperature drift requirement is met; over the course of almost 6 hours, the critical optical temperatures of the corner mirror and FTS changed less than half a degree. Thus there is no problems seen in any of the Optics Box temperatures, and it ran nominally for this stage.

The Electronics Box temperatures were cooler than expected. This was due to the initial temperatures for this stage of the flight; the temperature shock of the ascent had a larger effect than was expected. Also, in comparison to the Optics Box, there is less insulation and is larger; as a result, this box would be cooled faster through both convection across the back plate and conduction in the baseplate. However, when the heaters powered on and the instrument reached the warmer stratosphere, the temperatures steadied. Something to note was the temperature setpoint of the temperature controllers was higher than the actual temperatures, above 10°C. Through most of the flight, the temperatures were slightly below this. It was found that the setpoint of the temperature controller drifted, and as a result the amount of power sent to the heaters in the Electronics Box was lower than needed. The electronics temperature requirements were still met, but if this had happened to the temperature controller for the Optics Box, more serious issues could have occurred. A correction for this issue needs to be researched for future instruments, to ensure that temperatures stay nominal. The gondola baseplate temperature of roughly -27°C was in the range of simulated tests, and is good to know for future instrument simulations.

\subsubsection{Simulation}
With only conduction and radiation to include in the simulation, and with a steadier external environment, the simulations for this phase would be simpler. A total of 17 iterations were required to produce a model accurate to what was seen during flight. This was partly due to the initial temperatures from the previous simulation, which allowed an exact starting point for this simulation, following very precise initial temperatures compared to the flight. The conductive properties for this simulation were also all kept the same from what was determined from the TVAC tests and the ascent tests.

Much of the work of the radiation was already finished as well. The view factors were determined from previous simulations, and to maintain continuity between the simulations could not be changed. The only other value that needed to be determined was the ambient temperature. Due to the constant temperatures of both the instrument and the environment, they could be kept the same for the entire simulation, instead of attempting to determine a time curve. This made the iterations and determining the appropriate temperatures much easier. The majority if the ambient temperatures were taken from what was seen during the flight. 

With no convection to be determined, the only other aspect of the model that could be changed apart from the ambient temperatures was the power of the heaters. These were chosen, as with the TVAC test, from the measured instrument currents. More discussion into this is given in the full model discussion, in Section \ref{full_temp_model}. As with the ascent simulations, after a number of iterations, a thermal model was created that very closely matched the flight temperatures. Images of the final simulation are shown in Figure \ref{float_model}, and a comparison of the flight temperature curve to the simulated temperature curve is shown in Figure \ref{fig:float_temps_with_sims}.

\begin{figure}
    \centering
    \begin{subfigure}[h]{0.9\textwidth}
        \centering
        \includegraphics[width=\textwidth]{chap4_images/float_images/Test_16_BBEbox_FIXED.png}
        \caption{Thermal model of LIFE instrument at the end of the float phase, Blackbody Electronics Box view.}
        \label{fig:float_model_ebox}
    \end{subfigure}
    \begin{subfigure}[h]{0.9\textwidth}
        \centering
        \includegraphics[width=\textwidth]{chap4_images/float_images/Test_16_Ebox_FIXED.png}
        \caption{Thermal model of LIFE instrument at the end of the float phase, Electronics Box view.}
        \label{float_model_ebox}
    \end{subfigure}
    \caption{LIFE thermal model at the end of the nominal float stage of the flight.}
    \label{float_model}
\end{figure}

\begin{figure}
    \centering
    \includegraphics[width=\textwidth]{chap4_images/float_images/Float_temps_with_sims_V2.png}
    \caption{Flight temperatures during the float phase, compared to the simulated temperature curves after model updates. The corresponding actual and simulated temperature data is the same colour, with the simulated data shown dashed.}
    \label{fig:float_temps_with_sims}
\end{figure}

In Figure \ref{fig:float_temps_with_sims} the model now matches the actual temperatures very well. All temperatures match with a maximum error of 1°C, and the error on critical components less than 0.2°C. Once again, the final temperatures of this stage were very carefully simulated to match the actual temperatures as closely as possible, so the initial temperatures of the sunrise simulation match as closely as possible. Thermal loads for this simulation as well as the iterated thermal values can be found in the appendix.

\subsection{Sunrise}
The final part of the flight, known as the \textit{sunrise} stage, took place from 10:30 UTC to when the instrument was turned off prior to the descent, at 14:30 UTC. As mentioned previously this stage was not expected to occur, however there was difficulty in finding a landing zone for the gondola and the descent was delayed. While the instrument was originally planned to be turned off, it was decided that this was an opportunity to see how the instrument operated thermally in the sunlight. As there was time to create the thermal model for the sunlight and data was saved, this could be added to the atmospheric instrument thermal model.

\subsubsection{Measured Temperatures}
This stage of flight began as soon as temperatures began to increase after sunrise. They steadily rose throughout the remainder of the flight. While temperatures would be expected to begin levelling out again at some point, it is not surprising that the temperatures increased so quickly. The sun at high altitudes has a dramatic thermal effect, and no fans in the instrument were operating. There were also no vents anywhere either, as these were all enclosed to prevent any damage or intake to the inside of the boxes during the ascent. As a result the electronics steadily increased in temperature until they began to reach maximum allowable temperatures, and the instrument was turned off. A plot of these temperatures for this stage is shown in Figure \ref{full_temp_model}.

\begin{figure}
    \centering
    \includegraphics[width=\textwidth]{chap4_images/sunrise_images/Sunrise_temps_no_sims_V2.png}
    \caption{Flight temperatures as the sun rose and shone on the instrument, up until the end of the flight.}
    \label{fig:sunrise_temps_no_sims}
\end{figure}

The first aspect of the figure to point out is the drop around 13:00 UTC. This was due to an initial shutoff of the instrument, when the descent was expected to begin. When it was determined that the flight would continue for at least another hour after, the instrument was turned back on to take more images as well as to gather temperature data. However this drop does show the effect of turning off the electronics, and the amount of power that they generate. Most temperatures dropped by at least 2°C in just a few minutes. When power was restored the temperatures increased again quickly. 

As with the previous stages of the flight, there is a large difference in behaviour between the electronics boxes and the Optics Box. With the sun shining directly on the rear plate of the Electronics Box, temperatures climbed very quickly in the last few hours of flight. However, even with the same sun shining on the Optics Box, the temperatures remained very steady until the last few hours. This shows that the outer radiation plates were operating as expected; while the outer plate was absorbing the flux of the sun, very little of that heat was being transferred to the inner box, and then to the optics. Only after the outer box was heated for a considerable amount of time did the effect of sun begin to show on the interior components. This shows that if the instrument was expected to operate in daylight, the addition of similar radiation plates to sensitive areas of other boxes (the top of the Electronics Box for example, where the temperature-sensitive BMXS board is mounted) would allow the instrument to be used without issue. In addition it is noted that for most daylight flights extra shielding is used over the gondola to mitigate sun-exposure. This may have helped maintain the LIFE temperatures in their required range even with no other changes to the instrument. Finally, the temperature of the gondola baseplate is noted; this shows how quickly the temperature of the gondola itself rose, and that the rise in temperature of LIFE was expected.

\subsubsection{Simulation}
This simulation does not have the quick changes in temperature and environment of the ascent, nor does it have convection. However the environment is not nearly as steady as the float portion of the flight. Most importantly, the effect of the sun must be included in this simulation. As with the float, the conductive properties and initial temperatures are already well known. The main properties that must be iterated through for the final stage is the solar flux effect and the ambient temperature of the radiation.

The heat flux from the sun was more difficult to simulate than expected. The solar flux from the sun is well known in orbit to be 1400 $\mathrm{W/m^2}$. However, this heat flux decreases through the atmosphere, as some of this flux is absorbed by atmospheric molecules. Beyond this, the sun was not shining directly on the instrument for the entire stage of the flight, nor with its full intensity. Only part of the sun could sometimes be seen, or sometimes it may have been shining on the side of the gondola. Unfortunately adding the solar flux was not as easy as adding a 1400 $\mathrm{W/m^2}$ heat load to the side of the instrument. Through simulation iterations, it was eventually determined that a somewhat exponential flux curve led to the desired temperatures. Starting at 0 $\mathrm{W/m^2}$ for the beginning of the phase, it increased to 800 $\mathrm{W/m^2}$. It is believed that the sun did not shine directly on the instrument, and was warmed either through the wall of the gondola or through reflections. More data on the effect of sunlight is needed to verify this.

The final changes were made for the radiation ambient temperatures. It was similar to the previous simulations, which was changing the ambient temperatures to what was seen from the flight data. In addition to this, the ambient temperature was increased for the outer parts of the instrument that may have been warmed by the sun or from parts of the gondola which were warmed. The ambient temperature of radiation for the top of the boxes as well as the sides was chosen to increase to upwards of 30°C by the end of the flight.

\begin{figure}
    \centering
    \begin{subfigure}[h]{0.8\textwidth}
        \centering
        \includegraphics[width=\textwidth]{chap4_images/sunrise_images/Test_11_BBEbox_FIXED.png}
        \caption{Thermal model of LIFE instrument at the end of the entire flight, Blackbody Electronics view.}
        \label{fig:sunrise_model_ebox}
    \end{subfigure}
    \begin{subfigure}[h]{0.9\textwidth}
        \centering
        \includegraphics[width=\textwidth]{chap4_images/sunrise_images/Test_11_Ebox_FIXED.png}
        \caption{Thermal model of LIFE instrument at the end of the entire flight, Electronics Box view.}
        \label{sunrise_model_ebox}
    \end{subfigure}
    \caption{LIFE thermal model at the end of the sunrise stage of the flight, shortly before descent.}
    \label{sunrise_model}
\end{figure}

\begin{figure}
    \centering
    \includegraphics[width=\textwidth]{chap4_images/sunrise_images/Sunrise_temps_with_sims_V2.png}
    \caption{Flight temperatures during the sunrise phase, compared to the simulated temperature curves after model updates. The corresponding actual and simulated temperature data is the same colour, with the simulated data shown dashed.}
    \label{fig:sunrise_temps_with_sims}
\end{figure}

After 16 iterations a satisfactory thermal model for this stage was created. Images of this model are shown in Figure \ref{sunrise_model}, and the comparison of the simulated temperatures to the actual temperatures is shown in Figure \ref{fig:sunrise_temps_with_sims}. A interesting note from the thermal simulations, the effect of the solar heating on the top of the Optics Box is very obvious, heating the top plate to upwards of 70°C. The rest of the box still maintains a much cooler overall temperature.

Overall the model simulates the sunrise well. There is some error around the shutoff of the instrument, and this is due to the time step of the simulation. The power off time is roughly 10 minutes, and the entire phase of the flight is four hours. It would take a very long time to be able to run the simulation with steps of 10 minutes, so a coarser step is chosen. As a result the power off time in the simulation is slightly longer than the actual time. However, the temperatures still match in this region reasonably well. The temperatures are within 1°C of the actual temperatures, and as such the model is deemed successful. As before, more information about the thermal loads and temperature values can be found in the appendix.

\subsection{Full model}\label{full_temp_model}
With all sections simulated separately, they can be brought together as one full model. A comparison of the actual and simulated temperature curves for the entire flight is shown in Figure \ref{fig:full_flight_temps_with_sim}. The simulated temperatures overall now match very well with what was measured during flight. However, all thermal loads used still need to be verified with a thermal model of another instrument. Still, this provides a starting point for future thermal simulations.

\begin{figure}
    \centering
    \includegraphics[width=\textwidth]{chap4_images/Flight_temps_full_flight.png}
    \caption{The actual and simulated temperature curves for the entire flight. The corresponding actual and simulated temperature data is the same colour, with the simulated data shown dashed.}
    \label{fig:full_flight_temps_with_sim}
\end{figure}

Another way that this model can be verified is by comparing the heater currents used in the simulation to what was seen during the flight. The power sent to the heaters was tweaked in the simulation to help match the flight temperatures, but it must be ensured that the power curves that were used still match the measured current during flight. A current curve for the Optics Box/Electronics Box heaters and a current curve for the Blackbody Electronics Box heater is shown in Figure \ref{fig:actual_sim_current}.

\begin{figure}
    \centering
    \begin{subfigure}[h]{0.9\textwidth}
        \centering
        \includegraphics[width=\textwidth]{chap4_images/Ebox_OBox_I_sim_and_actual_new.png}
        \caption{Actual and simulated current for the Optics Box and Electronics Box heater current.}
        \label{fig:actual_sim_current_obox_ebox}
    \end{subfigure}
    \begin{subfigure}[h]{0.9\textwidth}
        \centering
        \includegraphics[width=\textwidth]{chap4_images/BBEbox_I_sim_and_actual_new.png}
        \caption{Actual and simulated current for the Blackbody Electronics Box heater current.}
        \label{fig:actual_sim_current_bbebox}
    \end{subfigure}
    \caption{A comparison between the current curves of what was measured during flight, and what was input into the simulation.}
    \label{fig:actual_sim_current}
\end{figure}

A margin of error is required for each system, for different reasons. In the Blackbody Electronics Box, this is to take into account that the measured current is also being used to power the blackbody system, not just the heater. This is known to be roughly 1.2\,A, however it can vary depending on the external environment temperatures, but it could not be measured directly. An error is included of 0.5\,A, to take into account the oscillation that could occur from the operation of the blackbody system. With the other boxes, the heater power is also running a few smaller systems, which is not directly taken into account by the simulations, with a current of 0.3\,A. As with the blackbody system power, these could oscillate, so the error is included to account for this.

Overall, the simulated temperature curve matches, within error, what was measured during flight. The overall shape is expected, with most of the power needed at the beginning to counteract the steep drop in temperatures through the ascent, leveling out around the float, and slowly turning off as the sun begins to heat the instrument. With this helping to verify the thermal model, there is more confidence that the thermal values chosen in the simulation accurately model high-altitude atmospheric conditions. This model can be used as a baseline in other thermal models of similar atmospheric instruments, which will help to gain confidence in future models and ensure those instruments will survive the flight environment.

\section{Summary}
In this chapter, all post-flight progress was discussed. The campaign and the overall results of the flight were discussed first. The flight was a success, with all thermal requirements being met, and the instrument operating nominally. Many measurements were taken over the course of the 14 hour flight, and temperature data was measured that could be used to help verify the thermal model. The mechanical results were also discussed, with the instrument surviving the harsh landing well but with some damage. The damage reviewed, and next steps for the instrument are being determined.

The main aspect of this chapter was the creation and discussion of the flight temperature model. With the temperatures well simulated for a survival test of the float portion of the flight, a more detailed model of the entire flight would be created. The temperatures measured were used to create a thermal model of all stages of flight that would match the temperatures measured for the ascent, float, and sunrise phases. Each of these phases were created and simulated in turn in SolidWorks, using some data from the flight and some known information about the flight environment. However, there is very little research on some of the more specific thermal properties of the atmosphere, such as convection. This model was created to gain some insight on these properties, by iterating through estimations of these properties until the simulation matched the flight. It is hoped that in the future this model can be used for other atmospheric instrument flights and the model can be improved through more flight data.

